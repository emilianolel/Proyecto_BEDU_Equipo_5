\documentclass{article}
\usepackage[utf8]{inputenc}
\usepackage{mathrsfs}
\usepackage{abstract} 
\usepackage[spanish,es-noshorthands]{babel}
\usepackage[margin=2cm]{geometry}
\usepackage{enumitem}
\usepackage{amsmath,amsthm,amssymb} 
\usepackage{graphicx}
\usepackage{pdfpages}
\usepackage{gensymb}
\usepackage{comment}
\usepackage{longtable}
\spanishdecimal{.}
\usepackage{titling}
\usepackage{float}
\usepackage{tikz}
\usetikzlibrary{arrows,scopes}
\usepackage{xcolor}
\usetikzlibrary{calc}
\usepackage{mathrsfs}
%\newcommand{\U}[1]{\, \mathrm{#1}} %comando que formatea las unidades con el espaciamiento correcto y en romanas
\usepackage{calligra}
\DeclareMathAlphabet{\mathcalligra}{T1}{calligra}{m}{n}
\DeclareFontShape{T1}{calligra}{m}{n}{<->s*[2.2]callig15}{}
\usetikzlibrary{positioning}
\usetikzlibrary{shapes.geometric}
\usetikzlibrary{shapes.misc}
\usepackage{xcolor}
\usepackage{textcomp}
\usepackage{gensymb}
\usepackage{url}
\setlength{\parskip}{10px}
\usepackage{fancyhdr}
\pagestyle{fancy}
\setlength{\headheight}{20pt}



\title{\textbf{Análisis de Datos de Diabetes}}

\author{
  García, Juan Manuel\\
  \texttt{first1.last1@xxxxx.com}
  \and
  Herrera, José Emiliano\\
  \texttt{eherrera1331@gmail.com}
  \and
  Miramontes, Fred\\
  \texttt{lilolilol@hotmail.com}
  \and
  Román, Christopher\\
  \texttt{ferrobin34@gmail.com}
  \and
  Sánchez, Gabriel\\
  \texttt{178294@iberopuebla.mx}
  \and
  Sánchez, Ludim\\
  \texttt{first1.last1@xxxxx.com}
}



\date{\today}


\begin{document}
%\maketitle
\twocolumn

\twocolumn[
  \begin{@twocolumnfalse}
    \maketitle
    \vspace*{-1cm}
    \begin{center}\rule{0.9\textwidth}{0.1mm} \end{center}
    \begin{abstract}
    En esta práctica se identificaron los productos, partículas resultantes, de una colisión de un haz de protones con un gas de $SF_6$, mejor conocido como \emph{hexafloruro de azufre}. Dichos productos fueron los siguientes iones: $F^+$, $SF^+$, $S^+$, $SF_2^{++}$, $SF^+$, $SF^+_2$, $SF^+_3$, $SF^+_4$ y $SF^+_5$, resultando éste último el más abundate. Los iones anteriores fueron identificados a través de su \emph{Relación Masa/Carga}, sin embargo no se notó presencia de iones $SF_6$, esto puede deberse a su configuración electrónica poco estable. Las colisiones se llevaron a cabo en el \emph{acelerador lineal} de bajas energías ($1\ keV$ a $10\ keV$) que se encuentra en el edificio \emph{Tlahuizcalpan} de la Facultad de Ciencias de la Universidad Nacional Autónoma de México. Además, en ésta práctica se discute el funcionamiento de las partes del acelerador lineal, así como la razón de la ausencia del ión $SF_6$.
    \end{abstract}
    \begin{center}\rule{0.9\textwidth}{0.1mm} \end{center}
    \vspace*{1cm}
  \end{@twocolumnfalse}
]

\section{Introducción}

\onecolumn{
  \bibliographystyle{abbrv}
  \bibliography{bibliografia}
}

\end{document}

