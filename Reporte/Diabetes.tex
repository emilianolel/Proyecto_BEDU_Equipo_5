\documentclass{article}
\usepackage[utf8]{inputenc}
\usepackage{mathrsfs}
\usepackage{abstract} 
\usepackage[spanish,es-noshorthands]{babel}
\usepackage[margin=2cm]{geometry}
\usepackage{enumitem}
\usepackage{amsmath,amsthm,amssymb} 
\usepackage{graphicx}
\usepackage{pdfpages}
\usepackage{gensymb}
\usepackage{comment}
\usepackage{longtable}
\spanishdecimal{.}
\usepackage{titling}
\usepackage{float}
\usepackage{tikz}
\usetikzlibrary{arrows,scopes}
\usepackage{xcolor}
\usetikzlibrary{calc}
\usepackage{mathrsfs}
%\newcommand{\U}[1]{\, \mathrm{#1}} %comando que formatea las unidades con el espaciamiento correcto y en romanas
\usepackage{calligra}
\DeclareMathAlphabet{\mathcalligra}{T1}{calligra}{m}{n}
\DeclareFontShape{T1}{calligra}{m}{n}{<->s*[2.2]callig15}{}
\usetikzlibrary{positioning}
\usetikzlibrary{shapes.geometric}
\usetikzlibrary{shapes.misc}
\usepackage{xcolor}
\usepackage{textcomp}
\usepackage{gensymb}
\usepackage{url}
\setlength{\parskip}{10px}
\usepackage{fancyhdr}
\pagestyle{fancy}
\setlength{\headheight}{20pt}



\title{\textbf{Análisis de Datos de Diabetes}}

\author{
  García, Juan Manuel\\
  \texttt{first1.last1@xxxxx.com}
  \and
  Herrera, José Emiliano\\
  \texttt{eherrera1331@gmail.com}
  \and
  Miramontes, Fred\\
  \texttt{liloliol@hotmail.com}
  \and
  Román, Christopher\\
  \texttt{ferrobin34@gmail.com}
  \and
  Sánchez, Gabriel\\
  \texttt{178294@iberopuebla.mx}
  \and
  Sánchez, Ludim\\
  \texttt{sanchezludim.anel@gmail.com}
}



\date{\today}


\begin{document}
%\maketitle
\twocolumn

\twocolumn[
  \begin{@twocolumnfalse}
    \maketitle
    \vspace*{-1cm}
    \begin{center}\rule{0.9\textwidth}{0.1mm} \end{center}
    \begin{abstract}
    En esta práctica se identificaron los productos, partículas resultantes, de una colisión de un haz de protones con un gas de $SF_6$, mejor conocido como \emph{hexafloruro de azufre}. Dichos productos fueron los siguientes iones: $F^+$, $SF^+$, $S^+$, $SF_2^{++}$, $SF^+$, $SF^+_2$, $SF^+_3$, $SF^+_4$ y $SF^+_5$, resultando éste último el más abundate. Los iones anteriores fueron identificados a través de su \emph{Relación Masa/Carga}, sin embargo no se notó presencia de iones $SF_6$, esto puede deberse a su configuración electrónica poco estable. Las colisiones se llevaron a cabo en el \emph{acelerador lineal} de bajas energías ($1\ keV$ a $10\ keV$) que se encuentra en el edificio \emph{Tlahuizcalpan} de la Facultad de Ciencias de la Universidad Nacional Autónoma de México. Además, en ésta práctica se discute el funcionamiento de las partes del acelerador lineal, así como la razón de la ausencia del ión $SF_6$.
    \end{abstract}
    \begin{center}\rule{0.9\textwidth}{0.1mm} \end{center}
    \vspace*{1cm}
  \end{@twocolumnfalse}
]

\section{Introducción}
La diabetes es una enfermedad crónica en la que el cuerpo no es capaz de regular la cantidad de glucosa en la sangre. La glucosa en la sangre es la principal fuente de energía y ésta proviene de los alimentos. Con el transcurso del tiempo, el exceso de glucosa en la sangre puede causar problemas de salud, tales como enfermedades del corazón, daño a los nervios y enfermedad de los riñones.

Los principales factores de riesgo de la diabetes son:
\begin{itemize}
	\item \textbf{Peso:} Mientras más tejido graso, más resistentes serán las células a la insulina.
	\item \textbf{Inactividad:} Mientras menos actividad se realice, mayor será el riesgo. La actividad física ayuda a controlar el peso, utiliza la glucosa como energía y hace que tus células sean más sensibles a la insulina.
	\item \textbf{Antecedentes familiares:} El riesgo se incrementa si padres o hermanos tienen diabetes tipo 2.
	\item \textbf{Raza o grupo étnico:} Aunque no está claro por qué, personas de ciertos orígenes, como las personas negras, hispanas, los indígenas estadounidenses y asiático-americanas, corren un mayor riesgo.
	\item \textbf{Edad:} El riesgo aumenta con la edad. Esto puede deberse a que la actividad física es menor, se pierde masa muscular y se aumenta de peso a medida que envejeces. Sin embargo la diabetes tipo 2 también está aumentando entre los niños, los adolescentes y los adultos jóvenes.
	\item \textbf{Presión arterial alta:} Una presión arterial de más de $140/90$ milímetros de mercurio (mm Hg) implica un alto riesgo de desarrollar diabetes tipo 2.
	\item \textbf{Niveles anormales de colesterol:} Si se cuenta con niveles bajos de lipoproteínas de alta densidad o de colesterol ``bueno", el riesgo de desarrollar diabetes tipo 2 será mayor. Lo triglicéridos son otro tipo de grasas que se transportan en la sangre. Las personas con niveles altos de triglicéridos afrontan un riesgo elevado de padecer diabetes tipo 2. 
\end{itemize}

Desde el año 2000, la diabetes mellitus en México es la primera causa de muerte entre las mujeres y la segunda entre los hombres. En 2010, esta enfermedad causó cerca de 83,000 muertes en el país. Es por esto que decidimos hacer un análisis sobre este problema.


\section{Objetivos}

Teniendo en cuenta el contexto y la importancia del estudio de esta enfermedad se plantean las siguientes preguntas:
\begin{itemize}
	\item ¿Existe alguna relación entre la diabetes y el grosor de la piel?
	\item ¿Existe alguna relación de la diabetes con la edad?
	\item Observar si existe correlación entre los factores anteriormente mencionados y la diabetes.
	\item ¿Existe alguna relación entre el embarazo y la diabetes?
	\item ¿Qué factor de riesgo hay que tener más en cuenta para evitar contraer diabetes?
\end{itemize}


\section{Dataset}

El dataset utilizado contiene datos recopilados del \emph{National Institute of Diabetes and Digestive and Kidney Diseases} y se puede consultar en la siguiente liga \url{https://www.kaggle.com/mathchi/diabetes-data-set}. Cada registro contiene ciertos parámetros de mujeres de la India de, al menos 21 años. Los campos del dataset son los siguientes:

\begin{itemize}
	\item \textbf{Pregnancies:} Número de veces que se ha embarazado.
	\item \textbf{Glucose:} Concentración de glucosa en plasma a dos horas en un test oral de tolerancia a la glucosa. 
	\item \textbf{BloodPressure:} Presión sanguínea.
	\item \textbf{SkinThicness:} Grosor de la piel del tricep.
	\item \textbf{Insulin:} Suero de insulina.
	\item \textbf{BMI:} Índice de masa corporal.
	\item \textbf{DiabetesPedigreeFunction:} Diabetes pedigree function.
	\item \textbf{Age:} Edad.
	\item \textbf{Outcome:} 0 o 1.
\end{itemize}


\section{Limpieza de los datos}

En general, el dataset se encontraba muy limpio, a excepción de unos valores NaN.

\begin{figure}[H]
	\centering
	\includegraphics[width=0.9\linewidth]{dataset_sucio.png}
	\caption{Dataset con el que se va a trabajar.}%
	\label{fig:dataset_sucio}
\end{figure}

Estos valores NaN se sustituyeron por $0$, resultando:

\begin{figure}[H]
	\centering
	\includegraphics[width=0.9\linewidth]{dataset_limpio.png}
	\caption{Dataset ya ``limpio''}%
	\label{fig:dataset_limpio}
\end{figure}

Por último, al dataset se le agregó un nuevo campo llamado ``ComposicionCorporal'', denotado como ``cc'' en la ecuación [\ref{cc}], con variables categóricas que indican el nivel del índice de masa corporal, es decir:

\begin{equation}\label{cc}
	cc=
	\begin{cases}
		\text{Bajo, si      } BMI \leq 18.85\\
		\text{Normal, si    } 18.5 < BMI \leq 24.9\\
		\text{Elevado, si   } 24.9 < BMI \leq 29.9\\
		\text{Obesidad, si  } 29.9 < BMI
	\end{cases}
\end{equation}

Resultando:

\begin{figure}[H]
	\centering
	\includegraphics[width=0.9\linewidth]{dataset_completo.png}
	\caption{Dataset completo.}%
	\label{fig:dataset_completo}
\end{figure}


\section{Análisis Exploratorio}

Una vez se obtuvo el dataset [\ref{fig:dataset_completo}] procedemos a realizar un análisis exploratorio de los datos.

Se obtiene una descripción general de los campos para mujeres con diabetes, tabla [\ref{fig:desc_diabetes}], y mujeres \emph{sanas}, tabla [\ref{fig:desc_sana}].

\begin{figure}[H]
	\centering
	\includegraphics[width=0.9\linewidth]{desc_diabetes.png}
	\caption{Descripción del \emph{subdataframe} con los registros de las mujeres con diabetes.}%
	\label{fig:desc_diabetes}
\end{figure}

\begin{figure}[H]
	\centering
	\includegraphics[width=0.9\linewidth]{desc_sana.png}
	\caption{Descripción del \emph{subdataframe} con los registros de las mujeres \emph{sanas}.}%
	\label{fig:desc_sana}
\end{figure}

Lo primero que se observa es que, del total de la muestra, hay 500 mujeres ``sanas''\footnote{Entiéndase por sanas a mujeres que salieron negativas al examen de detección de la diabetes.} y 268 mujeres con diabetes detectada.

\begin{figure}[H]
	\centering
	\includegraphics[width=0.65\linewidth]{bar_diabetes.png}
	\caption{Gráfica de barras que muestra la incidencia de diabetes en la muestra de población tomada.}%
	\label{fig:bar_diabetes}
\end{figure}

Cómo se muestra en la figura [\ref{fig:bar_diabetes}], el número 0 representa a una mujer \emph{sana} y 1 a una mujer con diabetes.

Por otro lado, y hablando en términos generales de la muestra, se observa que la mayoría de los registros corresponden a mujeres jóvenes entre 21 y 36 años, figura [\ref{fig:edad_muestra}]. 

\begin{figure}[H]
	\centering
	\includegraphics[width=0.65\linewidth]{edad_muestra.png}
	\caption{Edad de mujeres de la muestra.}%
	\label{fig:edad_muestra}
\end{figure}

Graficando de nuevo la concurrencia de edades pero ahora separadas según su diagnóstico

\begin{figure}[H]
	\centering
	\includegraphics[width=0.65\linewidth]{edad_diabetes.png}
	\caption{Gráfica de la concurrencia de edades de mujeres con diabetes.}%
	\label{fig:edad_diabetes}
\end{figure}

\begin{figure}[H]
	\centering
	\includegraphics[width=0.65\linewidth]{edad_no_diabetes.png}
	\caption{Gráfica de la concurrencia de edades de mujeres sanas.}%
	\label{fig:edad_no_diabetes}
\end{figure}

Como se observa en la figura [\ref{fig:edad_diabetes}] la distribución de mujeres con diabetes según su edad parece tener una distribución distinta a la mostrada en la figura [\ref{fig:edad_no_diabetes}]. Para la primera gráfica [\ref{fig:edad_diabetes}], parece que los casos se distribuyen normalmente\cite{Distro_normal} y para la gráfica [\ref{fig:edad_no_diabetes}] se tiene lo que parece ser un decaimiento exponencial. Esto se debe a que, como ya se mencionó, la mayoría de las mujeres registradas son jóvenes.

Como se menciona en la introducción, uno de los factores de riesgo es la concentración de glucosa en sangre. Por esta razón se obtiene la siguiente gráfica de cajas que se muestra en la figura  [\ref{fig:glucosa_diabetes}].

\begin{figure}[H]
	\centering
	\includegraphics[width=0.65\linewidth]{glucosa_diabetes.png}
	\caption{Gráfica de cajas para la concentración de glucosa en sangre de mujeres con diabetes.}%
	\label{fig:glucosa_diabetes}
\end{figure}

Por otro lado se muestra la gráfica de cajas para mujeres \emph{sanas}, figura [\ref{fig:glucosa_sanas}].

\begin{figure}[H]
	\centering
	\includegraphics[width=0.65\linewidth]{glucosa_sanas.png}
	\caption{Gráfica de cajas para la concentración de glucosa en sangre de mujeres \emph{sanas}.}%
	\label{fig:glucosa_sanas}
\end{figure}

Nótese que, para las mujeres sanas, el promedio de la glucosa en sangres es de $110.705\pm24.74\ [mg/dl]$ mientras que  para las mujeres con diabetes el promedio de glucosa en sangre es $142.159\pm29.546\ [mg/dl]$. Sin embargo, algo interesante a resaltar es que, considerando las desviaciones estándar de ambas cantidades, hay mujeres que tienen la glucosa elevada y, sin embargo, no tienen diabetes, lo que quiere decir que, definitivamente, la concentración de diabetes en la sangre, a pesar de ser un factor de riesgo, no es un factor suficiente para el diagnóstico de la enfermedad. Esta hipótesis toma aún más fuerza cuando se observan las siguientes gráficas:

\begin{figure}[H]
	\centering
	\includegraphics[width=0.65\linewidth]{glucosa_bar_diabetes.png}
	\caption{Gráfica de barras de los niveles de glucosa las mujeres diagnosticadas con diabetes.}%
	\label{fig:glucosa_bar_diabetes}
\end{figure}

\begin{figure}[H]
	\centering
	\includegraphics[width=0.65\linewidth]{glucosa_bar_sanas.png}
	\caption{Gráfica de barras de los niveles de glucosa las mujeres \emph{sanas}.}%
	\label{fig:glucosa_bar_sanas}
\end{figure}

Para realizar las gráficas, figuras [\ref{fig:glucosa_bar_diabetes}] y [\ref{fig:glucosa_bar_sanas}] se agregó otra columna al \emph{dataframe} llamada ``TipoDiabetes'', denotado como ``td'' en la ecuación [\ref{td}], con variables categóricas que indican el diagnóstico esperado.

\begin{equation}\label{td}
	td=
	\begin{cases}
		\text{Normal, si      } glucosa < 100\\
		\text{Prediabetes, si   } 100 \leq glucosa < 125\\
		\text{Diabetes, si  } 125 \leq glucosa
	\end{cases}
\end{equation}

Resulta ser que $128$ mujeres \emph{sanas} deberían tener diabetes por su alto nivel de glucosa en la sangre, sin embargo salieron negativas en la prueba.

Otro resultado a destacar lo muestra la matriz de correlación, tabla [\ref{fig:corr}].

\begin{figure}[H]
	\centering
	\includegraphics[width=0.9\linewidth]{corr.png}
	\caption{Tabla de correlación.}%
	\label{fig:corr}
\end{figure}

Números cercanos a 1 indican una correlación lineal positiva entre los campos, números cercanos a -1 indican una correlación lineal negativa entre los campos y números cercanos a 0 indican poca o nula correlación.

Se puede notar que la concentración de glucosa en la sangre es guarda una correlación moderada con los demás campos. Esto quiere decir que, a pesar de no ser un factor de riesgo determinante en el diagnóstico de la diabetes, éste resulta aumentar mucho la probabilidad de esta enfermedad. De igual manera el índice de masa corporal, ``BMI'' en el dataset, muestra una estrecha correlación con los demás factores de riesgo. 

A continuación se muestran unas gráficas de la concentración de glucosa en la sangre con los demás campos de la tabla, diferenciando con color azul los casos de mujeres \emph{sanas} y en naranja los casos de mujeres con diabetes.

\begin{figure}[H]
	\centering
	\includegraphics[width=0.65\linewidth]{BloodPressure_Glucose.png}
	\caption{Gráfica de la presión sanguínea y la concentración de la glucosa en la sangre.}%
	\label{fig:BloodPressure_Glucose}
\end{figure}

\begin{figure}[H]
	\centering
	\includegraphics[width=0.65\linewidth]{Insulin_Glucose.png}
	\caption{Gráfica de la cantidad de insulina y la concentración de la glucosa en la sangre.}%
	\label{fig:Insulin_Glucose}
\end{figure}

\begin{figure}[H]
	\centering
	\includegraphics[width=0.65\linewidth]{Pregnancies_Glucose.png}
	\caption{Gráfica del número de embarazos y la concentración de la glucosa en la sangre.}%
	\label{fig:Pregnancies_Glucose}
\end{figure}

\begin{figure}[H]
	\centering
	\includegraphics[width=0.65\linewidth]{SkinThickness_Glucose.png}
	\caption{Gráfica del grosor de la piel y la concentración de la glucosa en la sangre.}%
	\label{fig:SkinThickness_Glucose}
\end{figure}

En las gráficas anteriores podemos observar claramente que existen regiones diferenciadas para valores de Glucosa y Presión Sanguínea, figura [\ref{fig:BloodPressure_Glucose}], Glucosa e Insulina, figura [\ref{fig:Insulin_Glucose}], Glucosa y número de embarazos, figura [\ref{fig:Pregnancies_Glucose}] y Glucosa y Grosor de la Piel, figura [\ref{fig:SkinThickness_Glucose}], en las que, las mujeres tienden a desarrollar la Diabetes, zonas naranjas.




\onecolumn{
  \bibliographystyle{abbrv}
  \bibliography{bibliografia}
}

\end{document}

